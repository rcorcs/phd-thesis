
\section{Code Factoring}

%Function merging and code factoring are different techniques for solving the
%same fundamental problem of duplicated code.
Code factoring is a related technique that addresses the same fundamental
problem of duplicated code in a different way.
%While the former works by merging similar functions, the latter works by
%factoring out duplicated code~\cite{loki04}.
%Instead of merging similar functions, code factoring works by factoring out
%duplicated code into separate functions~\cite{loki04}.
Code factoring can be applied at different levels of the program~\cite{loki04}.
Local factoring, also known as local code motion, moves identical instructions
from multiple basic blocks to either their common predecessor or successor,
whenever valid~\cite{knoop94,briggs94,loki04}.
Procedural abstraction %(or outlining)
finds identical code
that can be extracted into a separate function, replacing all replicated
occurrences with a function call~\cite{loki04,dreweke07}.

Procedural abstraction differs from function merging as it usually works on
single basic blocks or single-entry single-exit regions.
Moreover, it only works for identical segments of code, and every identical
segment of code is extracted into a separate new function.
Function merging, on the other hand, works on whole functions, which can be
identical or just partially similar, producing a single merged function.

However, all these techniques are orthogonal to the proposed optimization and
could complement each other at different stages of the compilation pipeline.

